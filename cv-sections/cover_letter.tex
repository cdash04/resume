%%%%%%%%%%%%%%%%%%%%%%%%%%%%%%%%%%%%%%%%%
% Awesome Cover Letter
% XeLaTeX Template
% Version 1.1 (9/1/2016)
%
% This template has been downloaded from:
% http://www.LaTeXTemplates.com
%
% Original authors:
% Claud D. Park (posquit0.bj@gmail.com)
% Lars Richter (mail@ayeks.de)
% With modifications by:
% Vel (vel@latextemplates.com)
%
% License:
% CC BY-NC-SA 3.0 (http://creativecommons.org/licenses/by-nc-sa/3.0/)
%
% Important note:
% This template must be compiled with XeLaTeX, the below lines will ensure this
%!TEX TS-program = xelatex
%!TEX encoding = UTF-8 Unicode
%
%%%%%%%%%%%%%%%%%%%%%%%%%%%%%%%%%%%%%%%%%

%----------------------------------------------------------------------------------------
%	PACKAGES AND OTHER DOCUMENT CONFIGURATIONS
%----------------------------------------------------------------------------------------

\documentclass[11pt, letterpaper]{awesome-cv} % A4 paper size by default, use 'letterpaper' for US letter

\geometry{left=2cm, top=1.5cm, right=2cm, bottom=2cm, footskip=.5cm} % Configure page margins with geometry
 
\fontdir[fonts/] % Specify the location of the included fonts

% Color for highlights
\colorlet{awesome}{awesome-concrete} % Default colors include: awesome-emerald, awesome-skyblue, awesome-red, awesome-pink, awesome-orange, awesome-nephritis, awesome-concrete, awesome-darknight
%\definecolor{awesome}{HTML}{CA63A8} % Uncomment if you would like to specify your own color

% Colors for text - uncomment and modify
%\definecolor{darktext}{HTML}{414141}
%\definecolor{text}{HTML}{414141}
%\definecolor{graytext}{HTML}{414141}
%\definecolor{lighttext}{HTML}{414141}

\renewcommand{\acvHeaderSocialSep}{\quad\textbar\quad} % If you would like to change the social information separator from a pipe (|) to something else

%----------------------------------------------------------------------------------------
%	PERSONAL INFORMATION
%	Comment any of the lines below if they are not required
%----------------------------------------------------------------------------------------

\name{Christophe}{Duchesne-Ashworth}
\address{84 rue Demers, Québec, Qc, Canada, G1K 2A1}
\mobile{(+1) (581) 748-9626}

\email{christophe.duchesne-ashworth.1@ulaval.ca}
%\homepage{www.posquit0.com}
\github{cdash04}
\linkedin{Christophe-duchesne-ashworth}
%\skype{skypeid}
%\stackoverflow{SOid}{SOname}
%\twitter{@twit}

\position{} % Your expertise/fields
%\quote{``Make the change that you want to see in the world."} % A quote or statement

%----------------------------------------------------------------------------------------
%	RECIPIENT/POSITION/LETTER INFORMATION
%	All of the below lines must be filled out
%----------------------------------------------------------------------------------------

\recipient{Expert Sea}{1365 Avenue Galilée \#100 \\ Québec, QC G1P 4G4} % The company being applied to

\letterdate{\today} % The date on the letter, default is the date of compilation

\lettertitle{Poste de développeur junior} % The title of the letter

\letteropening{Chère M./Mme, \\ j'ai l'honneur de vous soumettre ma candidature pour le poste de développeur junior. } % How the letter is opened

\letterclosing{Sincèrement,} % How the letter is closed

% \letterenclosure[Joint]{Curriculum Vitae} % Any enclosures with the letter

\makecvfooter{\today}{Christophe D-Ashworth~~~·~~~Lettre de motivation}{} % Specify the letter footer with 3 arguments: (<left>, <center>, <right>), leave any of these blank if they are not needed
  
%----------------------------------------------------------------------------------------

\begin{document}

\makecvheader % Print the header

\makelettertitle % Print the title

%----------------------------------------------------------------------------------------
%	LETTER CONTENT
%----------------------------------------------------------------------------------------

\begin{cvletter}

%------------------------------------------------

\vspace{10mm}

Je suis un finissant du baccalauréat en informatique de l'Université Laval, qui souhaite se trouver un emploi en tant que développeur junior au sein de votre entreprise. Je suis aussi ouvert à l'idée à l'opportunité de faire un stage cet été. 

Je souhaite fortement travailler pour Expert Sea, car vous avez une forte réputation en ce qui attrait la qualité et l'innovation de ce que vous concevez. En effet, j'ai travaillé et fait un stage de recherche au centre de recherche du GRAAL en apprentissage automatisé ou j'ai rencontré deux anciens de votre entreprise, Nicolas Garneau et Jean-Thomas Baillargeon. J'ai énormément appris auprès d'eux et depuis, j'ai développé une fixation sur la qualité du travail accomplie dans le développement logiciel. Cependant, je considère que ma formation universitaire reçue ne met pas assez l'accent sur la qualité du code. Je suis donc conscient que je ne suis qu’au début de ma courbe d'apprentissage et qu'il est donc important de travailler dans un milieu ou la qualité et l'innovation prime afin de pouvoir développer au maximum mon potentiel.

Je crois avoir la curiosité et motivation nécessaire afin de devenir un atout pour l'entreprise et contribuer à son innovation. Mon expérience en recherche m'a permis de développer un esprit critique qui me permet de sans cesse m'améliorer et de rapidement maitriser de nouveaux outils ou de nouvelles technologies. Je suis aussi doué en apprentissage automatisé et avec les réseaux de neurones profonds. Donc je peux participer activement au développement et à l'intégration des nouveaux outils développés. 

Vous trouverez mon CV ci-joint,
Au plaisir d'avoir de vos nouvelles,





\end{cvletter}

%----------------------------------------------------------------------------------------

\vspace{15mm}

\makeletterclosing % Print the signature and enclosures

\end{document}